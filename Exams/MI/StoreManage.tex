%\section{Gestion d'un magasin \hfill 14 pts.  \Clocklogo 1h10  }

On souhaite �crire un programme qui permet de g�rer un magasin vendant des produits �lectrom�nagers . Ce programme permet, entre autres, de saisir les produits, de cr�er des commandes et d'�tablir  des  factures.
  


Un  produit est d�fini par  :
\begin{itemize}
\item \textsf{reference :} la r�f�rence du produit (par exemple 12985);
\item \textsf{designation :} la d�signation du produit (par exemple Micro-onde);
\item \textsf{prixUnitaire : } le prix unitaire du produit (par exemple 9850 DA) ;
\end{itemize}

\begin{enumerate}
\vspace{.2cm}
\item D�finir la structure \textbf{\textsf{Produit}}. \hfill  \textbf{2 pts} 


\vspace{.25cm}
\item Un catalogue contient le nombre de produits et l'ensemble des produits existants dans le magasin. Proposez une structure pour le Type  \textbf{\textsf{Catalogue}}. \hfill \textbf{1 pt}

\vspace{.25cm}
\item \'Ecrire les deux fonctions \textbf{\textsf{saisir\_Produit}} et     \textbf{\textsf{saisir\_Catalogue}}. \hfill \textbf{2 pt} 

\vspace{.25cm}
\item  \'Ecrire une fonction \textbf{\textsf{chercher\_Reference}} qui prend en entr�e une r�f�rence d'un produit et un catalogue et qui retourne l'indice du produit correspondant dans le catalogue. Si le produit n'existe pas dans le catalogue, la fonction renvoie -1. \hfill \textbf{2 pts} 

\vspace{.25cm}
\item Une commande contient plusieurs lignes de commande. Chaque ligne de commande contient une r�f�rence d'un produit ainsi que sa quantit�. Une commande ne doit pas contenir plus de  20 lignes. D�finir la structure \textbf{\textsf{Commande}}.  \hfill 
\textbf{2pts}
  
  \vspace{.25cm}
\item   \'Ecrire une fonction \textbf{\textsf{saisir\_Commande}} qui prend en entr�e une commande et  un Catalogue et qui demande � l'utilisateur de saisir les r�f�rences des produits ainsi que leurs quantit�s.\hfill \textbf{2pts} 

 \underline{\textbf{\Pointinghand
  Remarque.}} 
\begin{itemize}
\item La fonction ne doit accepter que les r�f�rences des produits existants dans le catalogue.
\item La fonction ne doit accepter que des quantit�s  strictement sup�rieurs � 0.

\end{itemize}

\vspace{.25cm}
\item �crire une fonction \textbf{\textsf{afficher\_Facture}} qui prend en entr�e  une commande et un catalogue et affiche la facture � l'�cran. Pour chaque r�f�rence de produit command�, la fonction  affiche sa r�f�rence, sa d�signation,   son prix unitaire, sa quantit� dans la commande  et le prix total de la ligne. A la fin de la facture, la fonction affiche le montant total � payer. \hfill \textbf{2 pts}


\vspace{.25cm}
L'affichage de cette fonction doit ressembler celui montr� ci-dessous  :

\begin{verbatim}
------------------------------- Facture ----------------------------------
Reference	    Designation      Prix Unitaire       Quantite        Total      
45665         T�l�viseur       66500DA             3             199500DA 
12985         Micro-onde       9850DA              2              19700DA
87653         R�frig�rateur    86000DA             1              86000DA 
86509         Cafeti�re        6500DA              4              26000DA
-------------------------------------------------------------------------- 
                                                 Montant Total = 332200DA
\end{verbatim}

\vspace{.5cm}

\item \'Ecrire une fonction \textbf{\textsf{main}} qui permet de saisir un catalogue,  cr�er une commande puis d'afficher sa facture. \hfill\textbf{1pt}

\end{enumerate}
