\documentclass[a4paper,11pt]{article} 
\usepackage{listings}
\usepackage{url}
\usepackage{hyperref}
\usepackage{graphicx}
\usepackage{amssymb}					
\usepackage{marvosym}
\usepackage[latin1]{inputenc}    
\usepackage[french]{babel}
\usepackage[T1]{fontenc}
\usepackage{pstricks}
\usepackage{colortbl}
\usepackage{fancybox}
\usepackage{enumerate}
\usepackage{tablists}
\usepackage[dvips]{epsfig}
\usepackage{makeidx}
\usepackage{amsmath,amssymb}
\usepackage{stmaryrd}
\usepackage{xspace}
\usepackage{float}
\usepackage{url}
\usepackage{multicol}
\usepackage{boxedminipage}
\usepackage{listings}
\usepackage{aurical}
\usepackage{slashbox}
\usepackage{threeparttable}
\usepackage{fancyhdr}
\usepackage{algorithm,algorithmic}
\usepackage{pstricks,pst-tree}
\pagestyle{empty}               % On ne num�rote pas les pages
\usepackage{vmargin}            % red�finir les marges
\setmarginsrb{2cm}{2cm}{2cm}{2cm}{0cm}{0cm}{0cm}{0cm}
\newcommand{\pascal}{\textsf{Pascal}\xspace}
\usepackage{pifont}
\usepackage[french]{minitoc}
\catcode`\�=\active
\catcode`\�=\active
\def�{\og\ignorespaces}
\def�{{\fg}}	
\usepackage{color}
\usepackage{lastpage}
\usepackage{wasysym}
\usepackage{tikz-qtree}
\pagestyle{fancy}
\renewcommand{\headrulewidth}{0pt}
\fancyfoot[c]{}
%\fancyfoot[L]{\Letter\  \texttt{\url{Mohamed.Messabihi@gmail.com}}}
\fancyhead[L]{}
\fancyhead[R]{}
%\fancyhead[L]{\sectionname}
%\pagestyle{fancy}
\usepackage{ifsym}
\setcounter{tocdepth}{1}
\renewcommand \thesection  {}
%-----------------------
\usepackage{filecontents}

%#######################################################################################


\begin{document}
\parbox{\textwidth}{ \small \rule{\textwidth}{1.5pt} \begin{minipage}{.5\textwidth}
\begin{flushleft} \textbf{Auteur} : Mohamed Messabihi\\   \textbf{Mati�re} : Programmation et structures de donn�es\\  \textbf{Date}  18 Mai 2017 \\ \textbf{Dur�e}  1h30 \\ \end{flushleft}
\end{minipage} 
 \hfill \begin{minipage}{.08\textwidth} \begin{center}\includegraphics[width=1cm]{logo.jpg}\end{center} \end{minipage} \hfill \begin{minipage}{.4\textwidth} \begin{flushright} Universit� Abou Bakr Belka�d - Tlemcen\\ Facult� des Sciences \\1\up{�re} Ann�e MI \\ Semestre 2 \end{flushright} \end{minipage} \rule {\textwidth}{1.5pt}}
%################################################################################
\pagenumbering{arabic} \setcounter{page}{1}
\begin{center}
\section*{Examen final}
\rule{.7\textwidth}{.5pt}\\
\footnotesize{{\textsf {Aucun document n'est autoris�\\
Les solutions doivent �tre r�dig�es en \textbf{C} \\
Les appareils portables doivent �tre  �teints et pos�s sur le bureau du surveillant}}}
\rule{.7\textwidth}{.5pt}
\end{center}
 
\section{Affichage \hfill . pts.  \Clocklogo .' }
\input{Supprimer-Occ}
\section{Allers-retours \hfill 8 pts.  \Clocklogo 30'}
\input{Allers-retours}
%%\section{Affichage    \hfill 8 pts.    \Clocklogo  30'}  

Qu'affiche les deux programmes suivants  :  
\vspace{2mm}

\begin{minipage}{.47\textwidth}
\begin{center}
\begin{minipage}{.9\textwidth}
 %footnotesize
\lstset{language=C,numbers=left,  numbersep=5pt,framexleftmargin=5mm, %frameround=fttt, 
frame=trBL, %stepnumber=2  
rulesepcolor=\color{gray},% backgroundcolor=\color{fondalgo},
belowcaptionskip=1\baselineskip,   breaklines=true,
  %frame=L,
  xleftmargin=\parindent,   language=C,   showstringspaces=false,   basicstyle=\footnotesize\ttfamily,   keywordstyle=\bfseries\color{green!40!black},   commentstyle=\itshape\color{purple!40!black},   identifierstyle=\color{blue},   commentstyle=\color{gray},   stringstyle=\color{red} , 
  %caption={Programme Myst�re},
  label=evaluation}
\lstinputlisting{exo1_rat2017.c}
 \end{minipage}
 \end{center}
  \end{minipage}
  \hspace{2mm} 
  \begin{minipage}{.5\textwidth}

 
\begin{center}
\begin{minipage}{.92\textwidth}
 %footnotesize
\lstset{language=C,numbers=left,  numbersep=5pt,framexleftmargin=5mm, %frameround=fttt, 
frame=trBL, %stepnumber=2  
rulesepcolor=\color{gray},% backgroundcolor=\color{fondalgo},
belowcaptionskip=1\baselineskip,   breaklines=true,
  %frame=L,
  xleftmargin=\parindent,   language=C,   showstringspaces=false,   basicstyle=\footnotesize\ttfamily,   keywordstyle=\bfseries\color{green!40!black},   commentstyle=\itshape\color{purple!40!black},   identifierstyle=\color{blue},   commentstyle=\color{gray},   stringstyle=\color{red} , 
  %caption={Programme Myst�re},
  label=evaluation}
\lstinputlisting{exo1_rat2017.c}
 \end{minipage}
 \end{center}
  \end{minipage}
\section{Rectangles \hfill 12 pts.  \Clocklogo 1h}

On souhaite programmer un �diteur graphique qui permet de dessiner des rectangles sur un plan muni d'un rep�re cart�sien.

Un rectangle est d�fini par :
\begin{itemize}
\item un \textsf{point} sur le plan qui repr�sente son coin sup�rieur gauche ;
\item une \textsf{longueur} ;
\item une \textsf{largeur} ;
\item et une \textsf{couleur} qui peut �tre soir blanche, noire ou grise.
\end{itemize} 

 Un point \textsf{P}  est d�fini par un couple de r�els :
 \begin{itemize}
 \item  \textsf{x} appel� l'abscisse de \textsf{P} ;
\item   \textsf{y} appel� l'ordonn�e de \textsf{P}.
 \end{itemize}


\begin{enumerate}
%\vspace{.2cm}
\item D�finir la structure {\textsf{Rectangle}}.   \textbf{3 pts} 


%\vspace{.25cm}
%\item Un {\textsf{Graph}} contient un ensemble de Rectangles. D�clarer un \textbf{\textsf{Graph}}. \hfill \textbf{1 pt}

%\vspace{.25cm}
\item \'Ecrire une fonction {\textsf{saisir\_Rectangle}}. \textbf{2 pts} 

%\vspace{.25cm}
\item  \'Ecrire une fonction {\textsf{deplacer}} qui prend en entr�e un rectangle r et un point $p\prime$ et puis elle d�place le rectangle r vers le point $p\prime$.  \textbf{1 pt} 



%\vspace{.25cm}
\item  \'Ecrire une fonction {\textsf{zoomer}} qui prend en entr�e un rectangle et un coefficient de zoom. La fonction maintien le coin sup�rieur gauche du rectangle  puis elle agrandie ou elle rapetisse les dimensions du rectangle en fonction du coefficient du zoom.     \textbf{1 pt} 

%\vspace{.25cm}
\item  \'Ecrire une fonction {\textsf{sym�trique}} qui prend en entr�e
 un rectangle r et  retourne le rectangle sym�trique 	 r$\prime$  du r par par rapport son coin sup�rieur gauche.   \textbf{1 pt}   

%\vspace{.25cm}
\item  \'Ecrire une fonction {\textsf{centre}} qui prend en entr�e un rectangle r et retourne le point qui se trouve � son plein milieu.  \textbf{1 pt} 



%\vspace{.25cm}
\item  \'Ecrire une fonction {\textsf{inclut}} qui prend en entr�e deux  rectangles r1 et r2   et renvoie 1 si r2 est inclus dans r1, 0 sinon. \textbf{2 pts} 


%\item \'Ecrire une fonction \textbf{\textsf{main}} qui permet de saisir un catalogue,  cr�er une commande puis d'afficher sa facture. \hfill\textbf{1pt}

\end{enumerate}

%\input{Rectangles}

 
\vfill \hfill \emph{ \og Bon courage  \fg}
\end{document}
\endinput